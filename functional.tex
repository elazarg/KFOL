\section{Functional and Type-Theoretic Readings}
\label{sec:functional}

A useful way to view the system is as the combination of two orthogonal ingredients. The first is an MPST-like, linear protocol part: a single global prefix that fixes the order of moves, the owner of each move, and the visibility/independence constraints between moves. This is the ``how the interaction must happen'' piece, and it is very close to the usual story for global types and their projections.

The second ingredient is a many-truths outcome: instead of a single postcondition, the terminal of a formula is a $K$-indexed family of FO goals, one per participant. This is the ``what each party ultimately wants'' piece. MPST on its own does not tell you this — it gives you structure and safety, not per-role winning conditions. By putting the two together in the same object, we can say: ``here is a protocol that enforces this information pattern, and here are the (possibly incompatible) objectives of all the parties who follow it.''

The game-theoretic readings (forceability, coalition success, thresholds) then 
sit exactly at the junction of these two parts: they look at the 
protocol-constrained runs, and they test which components of the multi-headed 
outcome can actually be realized under those constraints. So the protocol half 
limits what strategies can do; the outcome half tells us which truth we are 
trying to extract from those limited strategies.

To make these ideas concrete, we now present four equivalent computational 
views of \KFOL expressions, each highlighting different aspects of the protocol 
and outcome structure.

\subsection{Procedural view: linear coroutine}

Evaluation of a \KFOL\ formula can be understood procedurally as a
single-threaded, linear walk over the prefix:

\begin{enumerate}
  \item start with an assignment $s$;
  \item when you reach $P_i x.\,\Phi$, you must obtain a value for $x$
        \emph{from player $i$}, subject to the current visible history;
  \item extend $s$ with $x := a$;
  \item proceed to the next binder;
  \item at the terminal $\langle \fo_0,\ldots,\fo_{K-1}\rangle$, evaluate
        each $\fo_j$ under the final assignment.
\end{enumerate}

There are two closely related ways to make the ``obtain a value'' step concrete:

\begin{description}
  \item[External, coroutine-style.] The formula is a generator that \emph{yields}
        at $P_i x.\,\Phi$ to an external owner for player $i$, together with the
        visible history (all previously bound variables, minus those in the
        slash-set). Control resumes when that owner returns a value. This matches
        the ``several stateful parties are talking'' intuition.

  \item[Internal, pure-strategy-style.] The formula is evaluated against a fixed
        strategy profile $\profile = (\profile_0,\dots,\profile_{K-1})$. At
        $P_i x.\,\Phi$ we \emph{call} the pure function
        $\profile_i(\text{current-prefix}, s)$ to obtain $a$. This keeps everything
        inside the semantic function and is the version we use to define
        $\llbracket \Phi \rrbracket_{\mathcal M, s, \profile}$.
\end{description}

Both readings are linear (no backtracking, no branching); they differ only in
whether the ``player move'' is modeled as an external interaction or as a call to
a known pure strategy. The object language is the same in both cases; only the
operational account changes.

\begin{remark}[Inductive vs.\ coinductive readings]
The two operational accounts above are dual in the usual
inductive/coinductive sense. In the internal, pure-strategy reading we
assume all ingredients (structure, assignment, strategy profile) are given and
we evaluate by structural recursion on the prefix; this is an inductive
``fold over the game.'' In the external, coroutine-style reading we instead
expose the next required move and wait for the environment to supply a value;
here the behaviour is driven from the outside, which is closer to a
coinductive process view. Both target the same linear prefix; a systematic
comparison is left open.
\end{remark}

\subsection{Let-expression style: functional programs}
We can render the same evaluation as a nested functional program:

\begin{verbatim}
(* P0 x. P1 y. <phi0, phi1> *)
let
  val x = player_0 ()          (* first move: empty history *)
  val y = player_1 (x)         (* second move: sees x *)
in
  (phi0 x y, phi1 x y)
end
\end{verbatim}

Independence-friendly visibility is just ``drop the hidden variables from the call'':

\begin{verbatim}
(* P0 x. P1 y \ {x}. P2 z. <...> *)
let
  val x = player_0 ()
  val y = player_1 ()          (* / {x} => y does NOT see x *)
  val z = player_2 (x, y)      (* later move can see both *)
in
  terminal (x, y, z)
end
\end{verbatim}

In general, for a prefix
\[
  P_{0} x_0.\; P_{1} x_1 \setminus V_1.\; \dots.\; P_{n-1} x_{n-1} \setminus V_{n-1}.\; \langle \vec\fo \rangle
\]
we get the schematic program
\begin{verbatim}
let
  val x0 = player_0 ()
  val x1 = player_1 (visible_from [x0] minus V1)
  ...
  val xn = player_n (visible_from [...] minus Vn)
in
  (phi0 (x0,...,xn), ..., phi_{K-1} (x0,...,xn))
end
\end{verbatim}
A \KFOL\ game is thus an open program with one hole per player move; a strategy profile is the tuple of functions that fill those holes.

\subsection{Gallina-style: player-indexed dependent presentation}
We can write the same shape in a Gallina-like pseudocode that makes player ownership and capture explicit:

\begin{verbatim}
Definition game :=
  (* P0 move: sees nothing -> () *)
  let a := P0 (_ ()) in

  (* P1 move: sees a -> (a) *)
  let b := P1 (_ (a)) in

  (* P0 again, but / b: only sees (a, ()) *)
  let c := P0 (_ (a, ())) in  (* b hidden *)

  (* P2 move: sees everything so far *)
  let d := P2 (_ (a, b, c)) in

  {|
    when (a = b)             : P0;
    when (b <> c)            : P1;
    when (a = b \/ a <> b)   : P2
  |}.
\end{verbatim}

Interpretation:
\begin{itemize}
  \item each line \texttt{let x := Pi (\_ (history)) in} is a player-owned move with a strategy hole \texttt{\_};
  \item the \texttt{history} tuple is exactly the visible prefix for that move (prefix minus the slash-set);
  \item the final record is a $K$-indexed outcome: one logical objective per player.
\end{itemize}

\paragraph{Curry--Howard reading.}
This looks like a partially constructed proof term with several heads. The outer \texttt{let} chain is the protocol skeleton. Each \texttt{Pi (\_ (history))} is a typed subgoal owned by player $i$, to be solved using only \emph{that} history. A strategy profile is the collection of inhabitants for all such subgoals. The difference from the usual CH story is that we do \emph{not} collapse to one \texttt{Prop}; we keep one per player.

\subsection{Core calculus: first-order, indexed, success-parametric}
For metatheory it is convenient to strip this down to a first-order core and make ``success'' a parameter.

\paragraph{Syntax.} Let terms and FO formulas be standard:
\[
t ::= x \mid f(t_1,\dots,t_n)
\]
\[
\alpha ::= R(\vec t) \mid t_1 = t_2 \mid \top \mid \bot
  \mid \alpha \wedge \alpha \mid \alpha \vee \alpha
  \mid \alpha \to \alpha \mid \neg \alpha.
\]
Game formulas:
\[
\Phi ::= \langle \alpha_0,\dots,\alpha_{K-1} \rangle
  \mid P_i x:\tau.\ \Phi
  \quad (i \in \{0,\dots,K-1\}).
\]

\paragraph{Formation.} With FO context $\Gamma$:
\[
\frac{\Gamma \vdash \alpha_0 \quad \dots \quad \Gamma \vdash \alpha_{K-1}}
     {\Gamma \vdash \langle \alpha_0,\dots,\alpha_{K-1} \rangle}
\qquad
\frac{\Gamma, x:\tau \vdash \Phi}
     {\Gamma \vdash P_i x:\tau.\ \Phi}.
\]
This gives a global, player-symmetric spec: a player-indexed prefix ending in a $K$-tuple of FO goals.

\paragraph{Evaluation.} Given a structure $\mathcal M$, an assignment $s$, and a strategy profile $\profile$,
\[
\llbracket \langle \alpha_0,\dots,\alpha_{K-1} \rangle \rrbracket_{\mathcal M,s,\profile}
  := \bigl(\Iverson{\mathcal M,s \models \alpha_0},\dots,\Iverson{\mathcal M,s \models \alpha_{K-1}}\bigr),
\]
\[
\llbracket P_i x:\tau.\ \Phi \rrbracket_{\mathcal M,s,\profile}
  := \llbracket \Phi \rrbracket_{\mathcal M, s[x \mapsto a], \profile}
  \quad\text{where } a := \profile_i(P_i x:\tau.\ \Phi, s).
\]
So evaluation always produces a $K$-tuple of booleans (or $\mathsf{True}/\mathsf{False}$).

\paragraph{Parametric success.}
On top of this core we \emph{do not} fix a single success notion. Instead we admit judgments of the form
\[
\mathcal M, s \Vdash_{\mathsf{Succ}} \Phi
\]
where $\mathsf{Succ}$ specifies how we quantify over strategies. One instance is \emph{forceability}. Other instances (possibility, coalition winning, thresholds) are definable over the same evaluation.

\paragraph{Separation.}
This separates:
\begin{enumerate}
  \item structure: what is a well-formed, player-indexed game;
  \item evaluation: given strategies, what outcome vector do we get;
  \item success: which outcome patterns we count as ``good''.
\end{enumerate}
Only (1) and (2) belong to the core; (3) is meant to be chosen per application.
