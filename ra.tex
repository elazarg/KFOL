\section{Strategies as Relations; Profiles and Composition as Joins}
\label{sec:ra-view}

This section recasts (deterministic, pure) strategies for a \KFOL\ prefix as
\emph{functional relations} over the variables they are allowed to see.  The
payoff is conceptual clarity and a compact algebra of composition: profiles
are natural joins; hiding is projection; $\alpha$-renaming and player
permutation are schema renamings; clocks and broadcasts are key-shaping
patterns; IF-style independence is a key-exclusion constraint; and several
solution concepts become anti-join/division queries.  We keep the mathematics
lightweight but precise enough that a fully formal account is routine.

We treat each binder as a \emph{typed column}, the visibility DAG as the
\emph{set of key attributes} allowed to influence that column, a pure strategy
as a \emph{deterministic fill-rule} (a function from key to value), and a
strategy profile as the \emph{natural join} of those per-binder tables.  In
short: \emph{shape first, behavior second}.  The game specification fixes the
shape (columns, types, who can depend on whom); strategies populate it.

Why is this worth doing?  Two reasons.  First, it gives an intuitive picture
that matches how we already think about the prenex script: we fill the row
from left to right, and each move can only look at certain earlier columns.
Second, it buys an algebra of composition: sequencing is a join on an
interface (optionally followed by projection), clocks and broadcasts are
shorthand for key-shaping, independence is a key-exclusion constraint, and
several semantic queries (feasibility, forceability, ``no profitable
deviation'' in the $\{0,1\}$ case) become small relational patterns.

\paragraph{What this \emph{is} and \emph{is not}.}
It \emph{is} a faithful, lightweight reading of our deterministic, prenex
fragment that can be made fully formal with routine RA machinery.  It is
\emph{not} a new semantics for arbitrary utilities, stochastic choices, or
concurrent branching---those require extra layers (aggregation, probability,
or event-structure concurrency) that we intentionally avoid here.

\paragraph{How to read the section.}
If you only need the dictionary, use \Cref{fig:ra-cheatsheet} as a
reference.  If you want to see why the dictionary is sound, skim the short
lemmas that follow: they show that (i) joining total, functional strategy
tables yields a single completed play, and (ii) the composition and
independence patterns behave as advertised.

\begin{figure}[t]
  \centering
  \small
  \setlength{\tabcolsep}{6pt}
  \renewcommand{\arraystretch}{1.2}
  \begin{tabular}{p{0.44\linewidth} p{0.50\linewidth}}
    \toprule
    \textbf{Game-calculus notion} & \textbf{DB/RA reading (intuition)} \\
    \midrule
    Binder $P_i x_k$ with type $D_{x_k}$ & A \emph{typed column} $x_k{:}D_{x_k}$ in the eventual row \\
    Syntactic (prenex) order & The order we fill columns left-to-right \\
    Visibility edge $x_j\to x_k$ & $x_j$ is in the \emph{key} for $x_k$ (allowed dependency) \\
    Slash / independence for $x_k$ & \emph{Key exclusion}: certain earlier columns are \emph{forbidden} from $x_k$'s key \\
    Player $i$ controls $x_k$ & Ownership: who supplies the value for column $x_k$ given its key \\
    Game specification (no strategies) & Schema + key policy: column types and allowed dependencies \\
    Pure strategy for $x_k$ & Deterministic \emph{fill-rule} (graph of a function) from key$(x_k)$ to $x_k$ \\
    Strategy profile & \emph{Natural join} of all per-binder strategy tables \\
    Play (deterministic case) & The \emph{single completed row} produced by the join \\
    Terminal objective for player $i$ & A Boolean predicate on the completed row (selection/filter) \\
    Sequential composition $\Phi;\Psi$ & Join on an \emph{interface} of shared columns, then (optionally) project to hide internals \\
    Clock (unit choke-point) & Add a dummy unit column at the boundary; require it in the next phase's keys \\
    Broadcast & Put all prior columns into initial keys of the next phase (wide join) \\
    $\alpha$-renaming / player permutation & Column rename / ownership relabel; schema isomorphism \\
    Equivalence up to DAG isomorphism & Same dependency structure between columns (schema isomorphism) \\
    Legal histories at $x_k$ & Reachable partial rows projected to key$(x_k)$ \\
    Feasibility (well-posedness) & There exist per-column fill-rules total on legal histories \\
    Forceability of player $i$ & Anti-join pattern: no opponent completion produces a row failing $i$'s predicate \\
    Pure Nash (binary outcomes) & ``No $0\!\to\!1$ unilateral improvement'' expressed as self-join + anti-join \\
    \bottomrule
  \end{tabular}
  \caption{\textbf{Relational cheat-sheet.} A compact correspondence between the
  game calculus and a database/relational-algebra reading.  Use this as a
  waypoint if you get lost in the details.}
  \label{fig:ra-cheatsheet}
\end{figure}

\paragraph{Notation.}
We write $D_x$ for the domain of variable $x$, $X\to Y$ for a (relational)
functional dependency, and use standard RA operators: natural join ($\Join$),
projection ($\pi$), selection ($\sigma$), renaming ($\rho$), and set
difference ($\setminus$).  Schemas are sets of attributes named by variables.
Fix a prenex prefix with binders $b_1,\dots,b_n$ producing variables
$x_1,\dots,x_n$ and a visibility DAG $E\subseteq\{(j{\to}k)\mid j<k\}$.

\begin{definition}[Keys, local strategy tables]
For each position $k$ let the \emph{key} (visible history) be
\[
  K_k \;\coloneqq\; \{\,x_j \mid (j{\to}k)\in E\,\}\subseteq \{x_1,\dots,x_{k-1}\}.
\]
A \emph{local strategy table} for $x_k$ is a relation
\[
  S_k \;\subseteq\; \prod_{x\in K_k}\! D_x \;\times\; D_{x_k}
  \qquad\text{with}\qquad K_k \to x_k.
\]
Intuitively, $S_k$ is the graph of a total function from visible histories to
the move for $x_k$ (totality is stated below).
\end{definition}

\begin{definition}[Legal histories and totality]
Let $H_1\coloneqq\{\bullet\}$ and, inductively for $k>1$,
\[
  H_k \;\coloneqq\; \pi_{K_k}\!\Big(\; \Join_{j<k} S_j \;\Big).
\]
We say $S_k$ is \emph{total on legal histories} if for every $h\in H_k$ there
exists a unique $v\in D_{x_k}$ with $(h,v)\in S_k$.
\end{definition}

\begin{definition}[Strategy profile as join]
Given local tables $S_1,\ldots,S_n$ (one per binder), the \emph{profile} is
the natural join
\[
  S \;\coloneqq\; \Join_{k=1}^n S_k \;\subseteq\; \prod_{k=1}^n D_{x_k}.
\]
\end{definition}

\begin{lemma}[Determinacy of play]\label{lem:determinacy}
If each $S_k$ satisfies $K_k\to x_k$ and is total on its legal histories,
then $S$ is a singleton relation over $\{x_1,\dots,x_n\}$.
\end{lemma}

\emph{Proof sketch.} Topologically sort $E$; functionality and totality
determine a unique value for $x_k$ once the predecessors are fixed.  \qed

\paragraph{Equivalence and normalization.}
Equivalence up to (i) player permutation, (ii) $\alpha$-renaming, and (iii)
visibility-DAG isomorphism coincides with schema isomorphism for the family
$\{S_k\}$ (rename attributes; re-order does not matter).  Thus, adopting the
relational representation turns “equality up to those symmetries” into
literal equality up to renaming.


\subsection{Composition as Join + Projection}\label{subsec:ra-composition}

Let $\Phi$ end with variables $X^\Phi=\{x_1,\dots,x_m\}$ and $\Psi$ begin with
$Y^\Psi=\{y_1,\dots\}$.  To \emph{sequentially compose} $\Phi;\Psi$ choose an
\emph{interface} $I\subseteq X^\Phi$ and rename attributes in the first wave
of $\Psi$ so that every initial key $K_{y_\ell}$ is a subset of $I$ and the
domains match.  Then set
\[
  S^{\Phi;\Psi} \;\coloneqq\; \pi_{\text{externals}}\!\big( S^\Phi \Join_I S^\Psi \big),
\]
where $S^\Phi$ and $S^\Psi$ are the joins of their local tables, and
$\text{externals}$ projects away hidden interface attributes if desired.

Two useful patterns are immediate:

\begin{itemize}
\item \textbf{Clock (unit choke-point).} Introduce a fresh $t:\mathbf{1}$ as
the last variable of $\Phi$ and include $t$ in every initial key of $\Psi$.
Joining on $t$ enforces sequencing without adding information.
\item \textbf{Broadcast.} Include \emph{all} of $X^\Phi$ in the initial keys
of $\Psi$; the first wave of $\Psi$ can see all prior moves.  This is just a
wide natural join.
\end{itemize}

\begin{proposition}[Soundness of sequential composition]
With domain-consistent interfaces, $S^{\Phi;\Psi}$ coincides with the
play-by-play semantics of first resolving $\Phi$ then $\Psi$ under the
visibility wiring.  \qed
\end{proposition}

\subsection{Independence as Key Exclusion}\label{subsec:ra-independence}

IF-style annotations are read as \emph{schema constraints}.

\begin{definition}[Slash/visibility constraints]
For a binder annotated $P_i x_k \setminus V$ (variable-level) or $P_i x_k / J$
(role-level), enforce:
\[
  \forall x_j\in V \;\; (x_j \notin K_k)
  \qquad\text{or}\qquad
  \forall j\in J \;\; (\mathrm{pl}(x_j)\in J \Rightarrow x_j \notin K_k).
\]
Totality remains relative to $H_k=\pi_{K_k}(\Join_{j<k} S_j)$, i.e.\ to the
\emph{reduced} keys.
\end{definition}

\begin{remark}[No-signalling-by-key]
If a variable is slashed from $x_k$, it does not appear in $K_k$; hence
$x_k$ cannot (directly) depend on it.  Allowing a slashed variable into
$K_k$ admits signalling; forbidding it prevents such channels.  \qed
\end{remark}

\subsection{Objectives and RA Queries}\label{subsec:ra-queries}

Let each player’s objective be a predicate
$W_i \subseteq \prod_{k=1}^n D_{x_k}$ (evaluate the terminal tuple there).
Given a profile $S$, player $i$ succeeds iff $\sigma_{W_i}(S)=S$.  Beyond
single-profile checks, several semantic questions over \emph{sets of}
strategies are naturally relational (assume finite domains for enumeration).

\paragraph{Feasibility.}
A game is feasible if there exists a family $\{S_k\}$ with each $S_k$ total on
$H_k$.  Algorithmically: compute $H_k$ bottom-up and check emptiness of the
anti-join $H_k \setminus \pi_{K_k}(S_k)$ for all $k$.

\paragraph{Forceability (worst-case guarantee).}
Let $\mathcal{P}_i$ be the relation enumerating admissible tables for player
$i$ (correct schemas; total on legal histories) and $\mathcal{P}_{-i}$ the
product for the others.  The set of forcing strategies for $i$ is
\[
  \mathcal{F}_i \;=\;
  \mathcal{P}_i \;\setminus\;
  \pi_{\text{id}(S_i)}
  \bigl(
    \sigma_{\neg W_i}( S_i \Join \mathcal{P}_{-i} )
  \bigr),
\]
i.e.\ those $S_i$ for which there is no opponent completion producing a losing
play.  This is an anti-join/selection pattern.

\paragraph{Pure Nash in the $\{0,1\}$ case.}
A profile $S$ (a singleton play under \Cref{lem:determinacy}) is
Nash iff for each $i$ there is no alternative $S'_i$ such that
$\sigma_{W_i}( S'_i \Join S_{-i})$ is nonempty while
$\sigma_{W_i}( S_i \Join S_{-i})$ is empty.  Again, a self-join plus
anti-join.

\subsection{Small Running Example}\label{subsec:ra-example}

Two binders: $x$ (player $A$) then $y$ (player $B$) with $K_y=\{x\}$.
Let $D_x=D_y=\{0,1\}$.  Tables:
\[
  S_x(x) \text{ with }\emptyset\to x,
  \qquad
  S_y(x,y) \text{ with } x\to y.
\]
Profile $S=S_x\Join S_y \subseteq \{(x,y)\}$.  If
$W_B=\{(x,y)\mid x=y\}$, then $B$’s success on the profile is the check
$\sigma_{W_B}(S)=S$.  Sequentially compose with a $\Psi$ whose first move
$z$ sees $y$: set $K_z=\{y\}$ and compute $S^{\Phi;\Psi} = \pi_{x,y,z}
(S \Join S_z)$; hiding $y$ is $\pi_{x,z}$.

\paragraph{Scope and complexity (brief).}
When the visibility hypergraph is acyclic, worst-case optimal joins (e.g.\
Yannakakis-style) evaluate profiles and composition in time linear in the
input plus output size.  For cyclic schemas, joins can blow up in the worst
case; feasibility/forceability remain \emph{queries} but inherit that cost.
These are standard RA complexity boundaries; nothing exotic is required here.

\medskip
\noindent
\emph{Summary.} Treating local strategies as functional relations keyed by the
visibility DAG makes profiles a natural join, hiding a projection, wiring a
renaming, sequencing a join on an interface, clocks/broadcasts key-shaping,
and IF-style independence a key-exclusion constraint.  This alignment is
faithful at the granularity of our deterministic, prenex fragment, and is
fully formalizable with routine RA machinery.

\paragraph{Generalising to mixed strategies.}
All of the relational scaffolding survives if we replace per-binder functions by
\emph{behavioural kernels}. Assuming perfect recall (each binder’s key contains
everything the player could have observed so far), a mixed strategy for the
binder producing $x_k$ is a conditional distribution
\(p_k(x_k \mid K_k)\) with \(\sum_{v\in D_{x_k}} p_k(v\mid h)=1\) for every
\emph{reachable} key $h\in H_k$. The profile then induces a distribution over
complete plays that factorises along the visibility DAG:
\[
  \Pr(x_1,\dots,x_n) \;=\; \prod_{k=1}^n p_k\!\big(x_k \,\big|\, x_{K_k}\big).
\]
In database terms, keep the same schema and keys; attach a weight column to each
local table, interpret \emph{join} as weight multiplication, \emph{selection} as
zeroing weights, and \emph{projection}/hiding as marginalization (summing
weights). Independence, clocks, broadcasts, and sequential composition remain
purely structural—they only shape keys—so they carry over unchanged.

\paragraph{Solution concepts with randomness.}
Objectives lift from Boolean predicates on rows to expected values under
\(\Pr\). Pure-strategy checks like “no $0\!\to\!1$ deviation” become expected
payoff inequalities: for player $i$, a mixed profile is Nash iff for every
(unilateral) behavioral deviation $\hat p_i$,
\(\mathbb{E}_{p_i,p_{-i}}[u_i] \ge \mathbb{E}_{\hat p_i,p_{-i}}[u_i]\).
Relationally, this is the same self-join over opponents’ kernels, but the
decision is an aggregate comparison rather than an emptiness test. If shared
private randomness is needed across a player’s binders, introduce a hidden seed
variable sourced once and included in that player’s later keys (the same
key-shaping trick as the clock/broadcast). Normalization constraints
\(\sum_{v} p_k(v\mid h)=1\) are required only on \emph{legal} histories $h$,
mirroring totality in the deterministic case.
