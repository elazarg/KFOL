\begin{abstract}
We introduce \KFOL, a symmetric $K$-player calculus extending first-order logic. Formulas evaluate to outcome tuples in $\{0,1\}^K$ rather than truth values. The core syntax is a minimal, prenex-style language with player-labeled quantifiers ($P_i x.\Phi$), and we extend it with IF-style binders ($P_i x \setminus V.\Phi$) to constrain information flow.

For $K=2$, two operations mediate between classical \FOL\ and \KFOL: \emph{abstraction} embeds classical prenex formulas as symmetric, one-hot (zero-sum) games, and \emph{adversarial projection} recovers classical truth values by selecting a player's viewpoint. We analyze the core symmetries---rotation (player swap) and complement (payoff negation)---and show that classical negation is an emergent property of the two-player, zero-sum fragment. We also give a framework for structural combinators that build complex games from simpler components without extending the core language.

Because IF-style binders describe who may see which move, we show that a linear \KFOL\ quantifier prefix can be read as a linear, non-branching multiparty session protocol. We present two translations to MPST-style global types and adopt the visibility-restricting one so that the $\setminus V$ constraints become explicit in the protocol.

Finally, we give game-theoretic semantics defined by structural recursion, an abstraction–projection round-trip, and a complexity analysis showing that the core is PSPACE-complete while the IF-extension is NEXPTIME-hard.

K-FOL offers a single game-semantic core in which (i) the standard 2-player FO evaluation game appears as the zero-sum fragment via abstraction/projection, and (ii) the fully cooperative, single-specification style of session/choreography checking appears as the all-goals-equal fragment. In that sense it unifies a classical and a protocol-verification reading. We call this an adversarial verification perspective because the same semantic machinery can talk about non-aligned players as well.
\end{abstract}
