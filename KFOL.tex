
\section{Core Calculus: Syntax and Semantics}
\label{sec:k-core}

\subsection{Syntax}

Fix a classical first-order signature $\Sigma=(S,\mathcal F,\mathcal R)$ (sorts, function symbols, and relation symbols) and an integer $K \in \mathbb{N}$. Formulas of \KFOL\ are generated by
\[
\begin{array}{rcll}
\Phi &\Coloneqq& \langle \fo_0,\ldots,\fo_{K-1}\rangle
  & \text{(terminal $K$-tuple)}\\[2pt]
     &\mid& P_i x.\,\Phi
  & \text{(player-$i$ binding)} \quad i\in\{0,\ldots,K-1\},
\end{array}
\]
where each $\fo_i$ is a classical first-order formula over $\Sigma$.

Thus every formula has the shape
\[
P_{i_1} x_1.\; P_{i_2} x_2.\; \cdots\; P_{i_n} x_n.\;
\langle \fo_0,\ldots,\fo_{K-1}\rangle,
\]
i.e. a finite sequence of labeled binders followed by exactly one terminal.

\paragraph{Free variables.}
A variable is free in $\Phi$ if it occurs free in some $\fo_i$ and is not bound in the prefix. A formula is \emph{closed} if it has no free variables.

\paragraph{Alpha-equivalence.}
$P_i x.\,\Phi \equiv_\alpha P_i x'.\,\Phi[x\mapsto x']$ for $x'$ fresh.

\paragraph{Roles and coherence.}
Each binding is owned by some player index $i$, and all occurrences of the same index are governed by a single strategy (coherence). The terminal $K$-tuple encodes objectives: player $i$ succeeds if $\fo_i$ holds under the final assignment.

\paragraph{Visibility order.}
In this base fragment, the visible information at a move is exactly the set of earlier moves in syntactic (left-to-right) order. Later extensions introduce explicit syntax for constraining the dependence/visibility relation.

\paragraph{Underlying visibility DAG.}
Each \KFOL\ formula induces a directed acyclic graph (DAG) of information flow:
the nodes are the binder occurrences in the prefix, and there is an edge from
one binder to any later binder that is permitted to see its value (i.e.\ not
excluded by a visibility/slash set). We regard two games as equivalent if they
coincide up to (i) permutation of player indices, (ii) $\alpha$-renaming of
bound variables, and (iii) isomorphism of this visibility DAG.

\subsection{Histories and Strategies}

Fix a structure $\mathcal M$ for $\Sigma$ with domain $|\mathcal M|$.

\begin{definition}[Pure strategies]
For a formula $\Phi$ and a player $i$, a \emph{pure strategy} $\profile_i$ is a function that, for every subformula $\Psi$ of $\Phi$ of the form $\Psi = P_i x.\,\Phi'$ and every assignment $s$ over the free variables of $\Psi$, returns a value $a \in |\mathcal M|$:
\[\profile_i(\Psi, s) \in |\mathcal M|\]
This function represents player $i$'s choice for $x$ given the visible history (encoded in $s$) at the moment of the choice. A \emph{strategy profile} is a $K$-tuple $\profile=(\profile_0,\ldots,\profile_{K-1})$.
\end{definition}


\subsection{Evaluation Semantics}
\begin{definition}[Semantics]
\label[definition]{def:k-semantics}
Let $\Phi$ be a \KFOL\ formula, $\mathcal M$ a structure, $s$ an assignment for the free variables, and $\profile$ a strategy profile.
The outcome
\[
\llbracket \Phi \rrbracket_{\mathcal M, s, \profile} \in \{0,1\}^K
\]
is defined by structural recursion, where $\Iverson{\cdot}$ maps true to 1 and false to 0:
\begin{align*}
\llbracket \langle \fo_0,\ldots,\fo_{K-1}\rangle \rrbracket_{\mathcal M, s, \profile}&:= \big(\Iverson{\mathcal M,s\models \fo_0},\ldots,\Iverson{\mathcal M,s\models \fo_{K-1}}\big)\\
\llbracket P_i x.,\Phi \rrbracket_{\mathcal M, s, \profile}&:= \llbracket \Phi \rrbracket_{\mathcal M, s[x\mapsto a], \profile},\quad\text{where } a = \profile_i(P_i x.,\Phi, s)
\end{align*}
For closed $\Phi$, we write
$\llbracket \Phi \rrbracket_{\mathcal M, \profile} := \llbracket \Phi \rrbracket_{\mathcal M, \emptyset, \profile}$.
\end{definition}

\subsection{Forceability}

\begin{definition}[Individual forceability]
\label[definition]{def:k-force}
For player $i \in \{0,\ldots,K-1\}$ define
\[
F_i(\Phi,\mathcal M,s)
:= \exists \profile_i\ \forall \profile_0 \cdots \forall \profile_{i-1}\ \forall \profile_{i+1} \cdots \forall \profile_{K-1}\;
\bigl(\llbracket \Phi \rrbracket_{\mathcal M, s, \profile}\bigr)_i = 1.
\]
The \emph{forceability vector}
\[
F(\Phi,\mathcal M,s) := \bigl(F_0(\Phi,\mathcal M,s),\ldots,F_{K-1}(\Phi,\mathcal M,s)\bigr)
\in \{0,1\}^K
\]
records which players can guarantee their objective.
\end{definition}

\begin{remark}[Adversarial reading]
The quantification in \Cref{def:k-force} treats all players $j \neq i$ as
potential adversaries of player $i$: their strategies are universally
quantified and are not restricted to those that would help them achieve
their own objectives. In particular, $F_i(\Phi,\mathcal M,s)$ depends only
on player $i$'s component of the outcome tuple and on the interaction
structure, not on the other components. If we were to replace all
$\fo_j$ for $j \neq i$ in the terminal with $1$ or with $0$, the value of
$F_i$ would be unchanged.
\end{remark}

\section{Symmetric Group Action}
\label{sec:symmetry}

Player indices are nominal; this is captured by an action of $S_K$.

\subsection{Permutation of Formulas, Strategies, and Outcomes}

\begin{definition}[Permutation action]
\label[definition]{def:permute-k}
For $\pi \in S_K$, define $\pi \cdot \Phi$ by:
\[
\begin{aligned}
\pi \cdot \langle \fo_0,\ldots,\fo_{K-1}\rangle
  &:= \langle \fo_{\pi^{-1}(0)},\ldots,\fo_{\pi^{-1}(K-1)}\rangle,\\
\pi \cdot (P_i x.\,\Phi)
  &:= P_{\pi(i)} x.\,(\pi \cdot \Phi).
\end{aligned}
\]
Define the action on strategy profiles and outcome vectors by
\[
(\pi \cdot \profile)_i := \profile_{\pi^{-1}(i)},
\qquad
\pi \cdot (v_0,\ldots,v_{K-1})
:= (v_{\pi^{-1}(0)},\ldots,v_{\pi^{-1}(K-1)}).
\]
\end{definition}

Because we kept the prefix order unchanged, the syntactic visibility relation (``all earlier moves'') is preserved under $\pi$.

\begin{theorem}[Semantic equivariance]\label{thm:k-equivariance}
For any $\pi \in S_K$, structure $\mathcal M$, assignment $s$, and strategy profile $\profile$,
\[
\llbracket \pi \cdot \Phi \rrbracket_{\mathcal M, s, \pi \cdot \profile}
= \pi \cdot \llbracket \Phi \rrbracket_{\mathcal M, s, \profile}.
\]
\end{theorem}

\begin{corollary}[Forceability equivariance]\label{cor:k-force-equiv}
For any $\pi \in S_K$ and $i$,
\[
F_{\pi(i)}(\pi \cdot \Phi, \mathcal M, s)
\iff
F_i(\Phi, \mathcal M, s).
\]
\end{corollary}

\begin{definition}[Adversarial projection]
For $i \in \{0,\dots,K-1\}$ the adversarial projection
$\AdvProj{i}{\Phi}$ of a $\KFOL$ formula $\Phi$ is the first-order
formula defined by
\[
\begin{aligned}
\AdvProj{i}{\langle \fo_0,\dots,\fo_{K-1}\rangle}
  &:= \fo_i,\\
\AdvProj{i}{P_i x.\,\Phi}
  &:= \exists x.\,\AdvProj{i}{\Phi},\\
\AdvProj{i}{P_j x.\,\Phi}
  &:= \forall x.\,\AdvProj{i}{\Phi} \quad (j \ne i).
\end{aligned}
\]
\end{definition}

\begin{lemma}[Soundness w.r.t.\ individual forceability]
For any structure $\mathcal M$, assignment $s$, player $i$, and
$\KFOL$ formula $\Phi$,
\[
\mathcal M,s \models \AdvProj{i}{\Phi}
\quad\iff\quad
F_i(\Phi,\mathcal M,s).
\]
\end{lemma}

\subsection{Outcome Classes}
\label{sec:outcome-classes}

Fix a background theory $T$.
For a terminal $\langle \fo_0,\ldots,\fo_{K-1} \rangle$ we distinguish:

\begin{itemize}
  \item \textbf{Zero-sum (one-hot).}
    $\sum_i \Iverson{\mathcal M,s \models \fo_i} = 1$ for all $\mathcal M,s$.
    This covers the classical 2-player fragment via abstraction–projection
    (see \Cref{sec:k-to-2}).

  \item \textbf{Jointly realizable.}
    $\fo_0 \wedge \cdots \wedge \fo_{K-1}$ is $T$-consistent.
    Players have distinct objectives, but they are not mutually exclusive.
    Forceability then tells us which of them can be guaranteed unilaterally.

  \item \textbf{Independent.}
    All $2^K$ Boolean combinations of $\{\fo_0,\ldots,\fo_{K-1}\}$ are
    $T$-consistent.
    This is the fully non-interfering case and admits tensor-style
    factorization when the arenas also decompose.

  \item \textbf{Conflicting.}
    $\exists i \neq j$ with $T \models \fo_i \rightarrow \neg \fo_j$.
    The zero-sum (one-hot) case is a special instance.

  \item \textbf{Constant.}
    Each $\fo_i$ is $T$-equivalent to $\top$ or to $\bot$.
\end{itemize}

Since \KFOL formulas have a single terminal, a formula inherits its terminal's class.

\begin{remark}
Zero-sum formulas are the natural target for embedding classical FOL (\Cref{thm:fol-retract}).
\end{remark}

\begin{remark}[No general determinacy]
\label{rem:no-determinacy}
For $K > 2$ and arbitrary terminals, the forceability vector $\mathcal{F}(\Phi,\mathcal{M},s)$ can be any element of $\{0,1\}^K$. 
In particular, multiple players may simultaneously force their objectives, or none may be able to do so.
\end{remark}

\subsection{Outcome-insensitive binders as cheap talk}
\label{sec:cheap-talk}
A bound variable $x$ in $\Phi$ is \emph{outcome-insensitive} if for all assignments $s,s'$ differing only on $x$ we have the same outcome vector $v(\Phi,s)=v(\Phi,s')\in\{0,1\}^K$. Intuitively, $x$ never affects payoffs/truth. Nevertheless, if some later binder can \emph{observe} $x$ (i.e.\ is not declared independent of $x$), $x$ functions as \emph{cheap talk}: it can coordinate which equilibrium is played among several that are already available.

\subsection{Algebraic Structure}
The general $K$-FOL calculus carries a small but useful family of
syntactic symmetries. They all come from two ingredients:
(i) the $S_K$-action that permutes player indices, and
(ii) a uniform componentwise complement on objectives.
For $K=2$ this specializes to the familiar \emph{rotation} together with
componentwise complement, which we will later use in the 2-FOL development.

\paragraph{Componentwise complement.}
We define a uniform, coordinatewise negation of objectives:
\[
\Comp{\langle \fo_0,\dots,\fo_{K-1} \rangle}
  := \langle \neg \fo_0,\dots,\neg \fo_{K-1} \rangle,
  \qquad
\Comp{P_i x.\,\Phi} := P_i x.\,\Comp{\Phi}.
\]
This operator $\CompOp$ is an involution, $\CompOp^2 = \ID_{\KFOL}$, and it is
$S_K$-equivariant:
\[
\pi \cdot \Comp{\Phi} = \Comp{(\pi \cdot \Phi)} \qquad (\pi \in S_K).
\]
Equivalently, $\CompOp$ is the unique uniform player-symmetric nontrivial
involution on outcome tuples.

\begin{remark}[Negation under cooperation]
If the outcome is \emph{cooperative}, then every player's adversarial projection sees exactly classical negation.
\end{remark}

\begin{remark}[No zero-sum preservation]
Componentwise complement does not preserve zero-sum games for $K>2$; as a simple counterexample, take a game with the constant outcome $(true, false, false)$, complementing it you get $(false, true, true)$
\end{remark}

\begin{remark}[No global De Morgan duality]
The complement $\CompOp$ preserves player roles and only flips objectives, so under adversarial projection it yields
\[
\AdvProj{i}{\Comp{P_i x.\,\Phi}}
  \equiv \exists x.\,\neg \AdvProj{i}{\Phi},
\]
which is in general not equivalent to
\[
\neg \AdvProj{i}{P_i x.\,\Phi} \equiv \forall x.\,\neg \AdvProj{i}{\Phi}.
\]
Thus $\CompOp$ is a purely algebraic, player-symmetric symmetry and not a logical negation.
\end{remark}
