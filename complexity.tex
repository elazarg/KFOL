
\section{Complexity of Forceability}
\label{sec:k-complexity}

We analyze the complexity of the forceability problem:
given a finite structure $\mathcal M$, a closed \KFOL\ formula $\Phi$,
and a player $i$, decide if $F_i(\Phi,\mathcal M)$ is true.
We assume $\mathcal M$ is fixed and the formula $\Phi$ is the input.

\subsection{The Broadcast Core is PSPACE-complete}

First, we observe that the core calculus (without $\setminus V$ binders)
is PSPACE-complete, by a standard reduction from Quantified
Boolean Formulas (QBF).

\begin{proposition}[PSPACE-completeness of Core]
\label[proposition]{prop:pspace-core}
For fixed $K \ge 2$, the forceability problem $F_i$
for the broadcast fragment (no $\setminus V$ binders)
over finite structures is PSPACE-complete.
\end{proposition}

\begin{proof}[Proof sketch]
Membership in PSPACE is by a standard alternating Turing machine.
We can evaluate $F_i(\Phi, \mathcal{M})$ by a recursive algorithm
that simulates the game.
For a binder $P_i x.\,\Phi'$, it existentially guesses $a \in |\mathcal{M}|$
and recurses. For $P_j x.\,\Phi'$ (where $j \ne i$), it
universally checks all $a \in |\mathcal{M}|$ and recurses.
Since the domain $|\mathcal{M}|$ is finite, this takes polynomial
space in the depth of the formula.

Hardness is by reduction from QBF. Let $\psi$ be a QBF
\[
  Q_1 x_1 \dots Q_n x_n.\; f(x_1, \dots, x_n)
\]
over Boolean variables. We build a $K=2$ formula $\Phi_\psi$
over a structure $\mathcal{M}$ with domain $\{0, 1\}$.
This is a \textbf{one-hot} (\textbf{zero-sum} or \textbf{conflicting}) game,
as defined in \Cref{sec:outcome-classes}.
\[
\Phi_\psi :=
  P_{q(1)} x_1.\;
  P_{q(2)} x_2.\; \dots\;
  P_{q(n)} x_n.\;
  \langle f, \neg f \rangle
\]
where $q(j) = 0$ if $Q_j = \exists$ and $q(j) = 1$ if $Q_j = \forall$.
(Assuming $K=2$; for $K>2$ we pad with $\top$).
The QBF $\psi$ is true if and only if player $P_0$ can
force their objective, i.e., $F_0(\Phi_\psi, \mathcal{M})$ holds.
Since QBF is PSPACE-complete, so is our problem.
\end{proof}

\subsection{IF-style Independence is NEXPTIME-hard}

When we add the IF-style binders, the complexity jumps.
The problem is now at least as hard as Dependency QBF (DQBF),
which is NEXPTIME-complete.

\begin{proposition}[NEXPTIME-hardness with IF-binders]
\label[proposition]{prop:nexptime-if}
For fixed $K \ge 2$, the forceability problem $F_i$
with $\setminus V$ binders is NEXPTIME-hard.
\end{proposition}

\begin{proof}[Proof sketch]
We reduce from DQBF. A DQBF instance consists of a formula
\[
  \forall y_1 \dots \forall y_n \; \exists x_1 \dots \exists x_m.\; f(\vec{y}, \vec{x})
\]
and for each $x_j$, a dependency set $D_j \subseteq \{y_1, \dots, y_n\}$.
The instance is true iff there exist Skolem functions $f_1, \dots, f_m$,
where each $f_j$ \emph{only} depends on the variables in $D_j$,
such that $\forall \vec{y}, f(\vec{y}, f_1(D_1), \dots, f_m(D_m))$ holds.

We build a $K=2$ formula $\Phi$ over $\mathcal{M}=\{0,1\}$.
Let $P_1$ be the $\forall$ player and $P_0$ be the $\exists$ player.
For each $x_j$, define its ``hidden set'' $H_j$ as the
variables $x_j$ is \emph{not} allowed to see:
$H_j := \{y_1, \dots, y_n\} \setminus D_j$.

The formula $\Phi$ is:
\[
\begin{aligned}
\Phi :=
  P_1 y_1.\; \dots\; P_1 y_n.\;
  & P_0 x_1 \setminus H_1.\; \\
  & P_0 x_2 \setminus H_2.\; \\
  & \dots \\
  & P_0 x_m \setminus H_m.\;
  \langle f, \neg f \rangle
\end{aligned}
\]
The DQBF instance is true if and only if $F_0(\Phi, \mathcal{M})$ holds.

($\Rightarrow$) If DQBF is true, the Skolem functions $f_j$ exist.
$P_0$'s strategy $\profile_0$ is defined as $\profile_0(P_0 x_j \setminus H_j, s) := f_j(s|_{D_j})$.
This strategy is valid because it is independent of $H_j$.
By construction, it forces $f$ to be true.

($\Leftarrow$) If $F_0(\Phi, \mathcal{M})$ holds, then a valid strategy
$\profile_0$ exists. The binder $P_0 x_j \setminus H_j$
\emph{forces} $\profile_0$ to be independent of $H_j$.
This means $\profile_0$'s choice for $x_j$ can only depend on
variables \emph{not} in $H_j$, which is exactly $D_j$.
This strategy $\profile_0$ \emph{is} the set of
Skolem functions $f_j(D_j)$ that witnesses the DQBF.
\end{proof}

\subsection{Coalitional Forceability}

Having $K > 2$ also introduces new, hard problems
not present in 2-player logic. We can ask if a
\emph{coalition} of players can all win together.

\begin{definition}[Coalitional Forceability]
Let $S \subseteq \{0,\ldots,K-1\}$ be a coalition.
$S$ can jointly force its objectives if
\[
F_S(\Phi,\mathcal M)
:= \exists \{\profile_i\}_{i \in S}\;
   \forall \{\profile_j\}_{j \notin S}\;
   \bigwedge_{i \in S} \bigl( \llbracket \Phi \rrbracket_{\mathcal M, \profile} \bigr)_i = 1.
\]
\end{definition}

\begin{proposition}[PSPACE-hardness of Coalitional Forceability]
\label[proposition]{prop:coalitional-pspace}
For any fixed $K \ge 2$, the coalitional forceability problem
$F_S$ is PSPACE-hard, even without $\setminus V$ binders.
\end{proposition}

\begin{proof}[Proof sketch]
The reduction is again from QBF. Let $\psi$ be a QBF as in
\Cref{prop:pspace-core}.
Let $S = \{0\}$ be the coalition, and let the opposing
coalition be $J = \{1, \dots, K-1\}$.
We map $\exists$ quantifiers to $P_0$ (the coalition) and
$\forall$ quantifiers to $P_1$ (one of the opponents).
\[
\Phi_\psi :=
  P_{q(1)} x_1.\; \dots\; P_{q(n)} x_n.\;
  \langle f, \top, \dots, \top \rangle
\]
where $q(j) = 0$ if $Q_j = \exists$ and $q(j) = 1$ if $Q_j = \forall$.
The other players $P_2, \dots, P_{K-1}$ have no moves and
automatically win (their objective is $\top$).
The QBF $\psi$ is true $\iff \exists \profile_0 \forall \profile_1$
(and $\forall \profile_2 \dots$) such that $(\llbracket \Phi \rrbracket)_0 = 1$.
This is exactly $F_S(\Phi_\psi, \mathcal{M})$ for $S=\{0\}$.
The hardness holds.
\end{proof}

\subsection{Discussion}

These three statements together characterize the computational complexity
of \KFOL:

\begin{itemize}
  \item The broadcast $K$-player fragment is PSPACE-complete
        (\Cref{prop:pspace-core})
  \item Adding IF-style independence increases hardness to at least NEXPTIME
        (\Cref{prop:nexptime-if})
  \item Coalitional forceability is PSPACE-hard even in the broadcast fragment
        (\Cref{prop:coalitional-pspace})
\end{itemize}

These results establish that \KFOL\ forceability is a computationally rich
problem that quantifies over strategies under role-indexed information
constraints.