
\section{IF-style binders for roles}
\label{subsec:if-binders}
We refine the player binder to carry an \emph{independence set}, in the style of
Independence-Friendly (IF) logic~\cite{HintikkaSandu1989}. The syntax becomes
\[
  \Phi \Coloneqq
  \langle \fo_0,\dots,\fo_{K-1} \rangle
  \;\mid\;
  P_i x / J.\,\Phi
\]
where $i \in \{0,\dots,K-1\}$ is the role making the move, $x$ is the chosen
object, and $J \subseteq \{0,\dots,K-1\}$ is a finite set of \emph{roles from
which this choice must be independent}. Informally,
\emph{``player $i$ chooses a value for $x$, and the future moves of all roles in
$J$ must not be allowed to depend on this choice.''}

\paragraph{Semantics.}
The semantics of this binder is identical to $P_i x.\,\Phi$,
but it adds a \emph{constraint on valid strategies}.
A strategy profile $\profile$ is \emph{valid} for $\Phi$ if,
for every binder $P_i x \setminus V.\,\Phi'$, the strategy
function $\profile_i$ is independent of $V$.

Formally, for any two assignments $s_1, s_2$ over the free
variables of $P_i x \setminus V.\,\Phi'$,
\[
\bigl(\forall v \notin V,\ s_1(v) = s_2(v)\bigr)
\implies
\profile_i(P_i x \setminus V.\,\Phi', s_1) = \profile_i(P_i x \setminus V.\,\Phi', s_2).
\]
(Note: The implication states that if two assignments agree on all
variables \emph{except} possibly for those in $V$, the strategy must
produce the same choice. This formally captures independence from $V$.)

The definitions of evaluation $\llbracket\cdot\rrbracket$ and forceability $F_i$
(\Cref{def:k-semantics} and \Cref{def:k-force})
are unchanged, but the quantification $\exists \profile_i$
is now restricted to \emph{valid} strategies.

\subsection{Syntax Sugar: Where-Clauses}
\label{subsec:where}
For convenience we allow binders to carry an explicit local condition, in the
style of refined MPST. A binder
\[
  P_i x : \psi.\,\Phi
\]
is required to use only variables that were bound earlier in the prefix and are
visible to player $i$ (i.e.\ not hidden from $i$ by a slash-set). Its meaning is
defined by desugaring to the core calculus as follows:
\begin{enumerate}[nosep]
  \item $\psi$ is conjoined to player $i$'s terminal component, expressing that
        $i$ is responsible for establishing it;
  \item for every other player $j$, every subformula of $j$'s terminal component
        that depends on $x$ is wrapped under $\psi \rightarrow (\cdot)$,
        expressing that $j$ may rely on $i$ having established $\psi$.
\end{enumerate}
Intuitively, a where-clause records at the point of the move which property the
owner promises to make true, and makes this promise available as an assumption
to later players.
