\section{Cryptographic Primitives: Capabilities, Axioms, and Relevance}
\label{app:crypto}

This appendix summarizes standard cryptographic primitives in a way that is
useful for reasoning about information flow and coordination. For each
primitive we list: (i) what it provides in standard terms; (ii) canonical
security notions (assumptions/axioms); (iii) an intuitive/abstract reading; and
(iv) potential relevance to \KFOL.

\paragraph{Notation.}
We write $i,j$ for principals/roles, $pk_i,sk_i$ for public/secret keys,
$c$ for commitments/ciphertexts, and $\Pi$ for public assertions/certificates.

\subsection{Commitment Schemes}
\textbf{What it provides (standard).}
Two-phase interface: $\mathsf{Com}(x;r)\to c$ (commit) and
$\mathsf{Open}(c;x;r)$ (reveal). Pedersen, hash-based, etc.

\textbf{Canonical security notions.}
\emph{Hiding} (computationally, $c$ leaks no information about $x$).
\emph{Binding} (no adversary can open $c$ to two distinct values).
\emph{Correctness} (honest opens verify).

\textbf{Intuitive / abstract meaning.}
Separation of \emph{declaration} from \emph{definition}: “fixed now, opaque
until chosen to reveal.” Lets you assert relations later (equality, membership)
without retroactively changing the value.

\textbf{Potential relevance to \KFOL.}
Model “causal without value” dependencies; schedule \emph{when} a value becomes
usable; link hidden choices across branches by later equalities.

\subsection{Zero-Knowledge Proofs (Commit-and-Prove)}
\textbf{What it provides (standard).}
Prover convinces verifier that a statement $\phi(w;z)$ is true about a hidden
witness $w$ (often committed) and public input $z$, revealing nothing beyond
validity (Sigma protocols, SNARKs/STARKs).

\textbf{Canonical security notions.}
\emph{Soundness} (false statements are unprovable);
\emph{Zero-Knowledge} (no leakage beyond truth);
optionally \emph{(Knowledge-)soundness} (a valid proof implies existence of a
witness, sometimes extractable in the ideal model).

\textbf{Intuitive / abstract meaning.}
“Declaration + properties”: you can publish any property of a hidden value
without revealing the value itself; selectively disclose.

\textbf{Potential relevance to \KFOL.}
Express rich constraints on hidden choices (range, membership, equality, functional
relationships) while preserving independence/visibility structure.

\subsection{Digital Signatures}
\textbf{What it provides (standard).}
$\mathsf{Sign}_{sk_i}(m)\to\sigma$, $\mathsf{Verify}_{pk_i}(m,\sigma)\in\{\top,\bot\}$.

\textbf{Canonical security notions.}
\emph{EUF-CMA unforgeability}; \emph{transferable verifiability};
(non-\emph{deniable}) \emph{non-repudiation}.

\textbf{Intuitive / abstract meaning.}
Attribution that survives transfer: anyone can check who endorsed what.

\textbf{Potential relevance to \KFOL.}
Attach statements to roles; make public assertions accountable; distinguish
authorized from informal claims.

\subsection{Public-Key Encryption (PKE)}
\textbf{What it provides (standard).}
$\mathsf{Enc}_{pk_j}(m)\to c$, $\mathsf{Dec}_{sk_j}(c)\to m$. Typically via KEM+AEAD.

\textbf{Canonical security notions.}
\emph{IND-CCA confidentiality} (best practice); authenticity only if combined with signatures.

\textbf{Intuitive / abstract meaning.}
\emph{Capability split}: public \emph{write} to $j$’s mailbox; private \emph{read}
by $j$ only. Adds a private edge to $j$ without informing others.

\textbf{Potential relevance to \KFOL.}
Implements private visibility / IF-style independence; enables hidden coordination
without disturbing public knowledge.

\subsection{Symmetric Encryption (AEAD)}
\textbf{What it provides (standard).}
Authenticated encryption with associated data: $\mathsf{Enc}_k(m,ad)\to c$,
$\mathsf{Dec}_k(c,ad)\to m/\bot$.

\textbf{Canonical security notions.}
\emph{IND-CPA/CCA confidentiality} and \emph{INT-CTXT integrity}; nonce-discipline assumptions.

\textbf{Intuitive / abstract meaning.}
A secure channel once a key exists; $ad$ binds context (session, role, epoch).

\textbf{Potential relevance to \KFOL.}
Implements reliable private subchannels after key agreement; preserves global
structure while enabling local secrecy.

\subsection{Hashes, Merkle Trees, and Vector Commitments}
\textbf{What it provides (standard).}
Binding, succinct handles for data; short membership/non-membership proofs.

\textbf{Canonical security notions.}
\emph{Collision resistance} (hashes); \emph{position-binding} and proof soundness (Merkle/VC).

\textbf{Intuitive / abstract meaning.}
Names for large structures; selective disclosure of parts.

\textbf{Potential relevance to \KFOL.}
Attach short certificates to public state; avoid flooding the public script when
only a property matters.

\subsection{Public Randomness: Beacons and VRFs}
\textbf{What it provides (standard).}
Common unpredictable randomness (\emph{beacon}); per-identity, privately sampled
but publicly verifiable randomness (\emph{VRF}).

\textbf{Canonical security notions.}
\emph{Unpredictability}, \emph{uniqueness}, \emph{bias-resistance}.

\textbf{Intuitive / abstract meaning.}
Stochastic moves that are either common-knowledge (beacon) or privately verifiable (VRF).

\textbf{Potential relevance to \KFOL.}
Model probabilistic binders without introducing side information; coordinate
random choices.

\subsection{Secret Sharing and Threshold Cryptography}
\textbf{What it provides (standard).}
$t$-of-$n$ reconstruction for secrets (Shamir); threshold signatures/decryption.

\textbf{Canonical security notions.}
\emph{$t$-privacy} ($< t$ learn nothing), \emph{$t$-correctness} ($\ge t$ succeed);
\emph{monotonicity}.

\textbf{Intuitive / abstract meaning.}
Fractional permissions / joint custody; coalitions gate learning or action.

\textbf{Potential relevance to \KFOL.}
Coalition-controlled reveals; “broadcaster/choke-point” nodes realized as threshold events.

\subsection{Secure Multiparty Computation (MPC)}
\textbf{What it provides (standard).}
Compute $y=f(x_1,\dots,x_n)$ while keeping inputs private; reveal only agreed outputs.

\textbf{Canonical security notions.}
\emph{Correctness}, \emph{privacy} against coalitions up to a threshold; optionally
\emph{robustness}/\emph{fairness}.

\textbf{Intuitive / abstract meaning.}
Compute on secrets; disclose only what the specification allows.

\textbf{Potential relevance to \KFOL.}
Publish complex functions of hidden choices without violating independence; keep value-visibility minimal.

\subsection{Append-Only Ledgers / Bulletin Boards}
\textbf{What it provides (standard).}
Public, tamper-evident, globally ordered log of statements.

\textbf{Canonical security notions.}
\emph{Append-only}; \emph{consistency} (common prefix); liveness under network assumptions.

\textbf{Intuitive / abstract meaning.}
A shared timeline; prevents backdating and equivocation.

\textbf{Potential relevance to \KFOL.}
Implements a globally agreed causal order for public assertions; supports time-based reasoning.

\subsection{Verifiable Delay Functions (VDFs)}
\textbf{What it provides (standard).}
Enforces a minimum delay $\Delta$ to learn $y=\mathsf{VDF}^\Delta(x)$ with a
short, quickly verifiable proof.

\textbf{Canonical security notions.}
\emph{Sequentiality}, \emph{uniqueness}, \emph{fast verification}.

\textbf{Intuitive / abstract meaning.}
Time-locks: information becomes available only after $\Delta$.

\textbf{Potential relevance to \KFOL.}
Model latency constraints (“reveal only after time $t$”) without adding new players.

\subsection{Anonymous Credentials; Ring and Group Signatures}
\textbf{What it provides (standard).}
Authorized statements with hidden author (ring); membership-certified anonymity with optional opening/revocation (group/credentials).

\textbf{Canonical security notions.}
\emph{Unforgeability}, \emph{anonymity}; (un)linkability depending on scheme; accountable opening for group sigs.

\textbf{Intuitive / abstract meaning.}
Authorization decoupled from identity; speak as a class without revealing who.

\textbf{Potential relevance to \KFOL.}
Model “someone with role-rights said this” without disturbing role anonymity.

\subsection{Oblivious Transfer (OT), Mixnets, and ORAM}
\textbf{What it provides (standard).}
OT: receiver learns one of many items, sender learns not which.
Mixnets: unlink sender/receiver; ORAM: hide access patterns to memory.

\textbf{Canonical security notions.}
Standard indistinguishability definitions against traffic/choice observers.

\textbf{Intuitive / abstract meaning.}
Hides meta-information (who talks to whom; which index was used).

\textbf{Potential relevance to \KFOL.}
If independence should hold even against traffic analysis, these support private edges without meta-leakage.

\bigskip
\noindent\textbf{Design takeaway.}
For most information-flow arguments one can assume the following \emph{axioms}
without committing to a concrete scheme:
\begin{itemize}
  \item \emph{Declaration vs.\ definition:} values can be fixed yet hidden until revealed (commitment).
  \item \emph{Selective disclosure:} properties of hidden values can be asserted without revealing the values (zero-knowledge).
  \item \emph{Capability split:} public write / private read channels exist (PKE); private authenticated channels exist given a key (AEAD).
  \item \emph{Attribution:} transferable accountability for statements (signatures).
  \item \emph{Coalition gates:} certain reveals/actions require threshold participation (secret sharing/threshold crypto).
  \item \emph{Order/time:} public append-only order and enforceable delays can be assumed (ledger, VDF).
\end{itemize}
These are sufficient to justify common \KFOL\ constructions around visibility,
independence, and coordination \emph{without} embedding cryptographic details in
the language.

\subsection{Payoff as Evidence of (In)Capability or Knowledge}
\label{subsec:payoff-ignorance}

A recurring theme across economics, learning theory, and cryptography is using
\emph{payoffs} (or scores) as operational evidence about an agent’s information
state or computational capability. The informal reading is: if an agent is
\emph{rational} (utility-maximizing) and nevertheless fails to achieve a certain
payoff threshold that would be attainable by someone with a specific piece of
knowledge or computational ability, then the observed shortfall is evidence that
the agent lacks that knowledge/ability. This idea appears under several names:

\paragraph{Rational (Interactive) Proofs and Utility Gaps.}
In \emph{rational proofs} (a.k.a.\ \emph{rational interactive proofs}), the
verifier designs payments so that a prover maximizes expected payoff if and only
if it supplies correct information (e.g., a correct computation). The key
notion is a \emph{utility gap}: there exists $\delta>0$ such that any strategy
producing the correct answer yields an expected payoff at least $\delta$ higher
than any incorrect strategy. Persistently earning less than the optimal payoff
is then formal evidence of \emph{incapability} (or irrationality). This is the
closest formalization of ``payoff $\Rightarrow$ (lack of) capability.''

\paragraph{Proper Scoring Rules and Peer Prediction.}
With \emph{strictly proper scoring rules}, truthful reports of one’s belief
uniquely maximize expected score. Systematically suboptimal scores thus serve as
evidence that the reporter \emph{lacks} accurate information (or chose to
misreport against interest). When ground truth is unavailable, \emph{peer
prediction} and \emph{Bayesian Truth Serum} engineer payoffs so that honest,
well-informed reports form (the unique) best response in equilibrium.

\paragraph{Signaling and Separating Equilibria.}
In \emph{costly signaling} (e.g., Spence), only high-ability or well-informed
types find it profitable to take a particular action. Failure to take the costly
signal is Bayesian evidence of \emph{not} being the high-ability/informed type,
again reading payoff-differences as evidence about capability.

\paragraph{Algorithmic Knowledge (Resource-Bounded Knowing).}
\emph{Algorithmic knowledge} models explicit knowledge via a knowledge algorithm:
an agent ``knows'' $\varphi$ precisely when its algorithm can derive $\varphi$
within given resources. Coupled with payoff design (as above), inability to
produce a derivation/witness that would strictly improve payoff is evidence of
non-knowledge under the stipulated resource bound.

\medskip
\noindent
\textbf{Remark on payoff granularity.}
The literature typically relies on \emph{numeric} payoffs or scores to create a
strict improvement margin (the utility gap $\delta$). In the current 
$\{0,1\}$-valued setting, such margins collapse: ``profitable deviation'' reduces
to turning a $0$ into a $1$, and you cannot use a \emph{larger} loss in one game
to certify the ability to secure a \emph{smaller} but positive payoff elsewhere.
If you later extend to graded payoffs, you can leverage cross-game arguments of
the form: ``accepting a higher loss here credibly certifies an ability that
secures a smaller gain there,'' which is the standard currency in scoring-rule
and rational-proof analyses.
