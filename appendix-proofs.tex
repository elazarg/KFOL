\section{Proofs}
\label[appendix]{app:proofs}

\begin{proposition}[Canonical 2-player embedding]
\label{prop:abs-unique}
Fix $i \in \{0,1\}$. Assume a 2-player calculus with binders $P_0 x.\,$ and $P_1 x.\,$ and an adversarial projection operation
\[
  \AdvOp_j : \text{2-FOL} \to \FOL \qquad (j=0,1)
\]
such that
\begin{align*}
  \AdvOp_i(\langle \psi_0,\psi_1\rangle) &= \psi_i, &
  \AdvOp_i(P_i x.\,\Phi) &= \exists x.\,\AdvOp_i(\Phi), &
  \AdvOp_i(P_{1-i} x.\,\Phi) &= \forall x.\,\AdvOp_i(\Phi),
\end{align*}
and similarly for $j=1-i$ with the roles of $\exists$ and $\forall$ swapped.
Then there is at most one operation
\[
  \AbsOp_i : \FOL \to \text{2-FOL}
\]
satisfying
\begin{enumerate}
  \item $\AdvOp_i \circ \AbsOp_i = \mathrm{id}_{\FOL}$,
  \item $\AdvOp_{1-i} \circ \AbsOp_i = \neg(\,\cdot\,)$,
  \item $\AbsOp_i$ is compositional over the FOL constructors (atoms, $\neg$, $\land$, $\lor$, $\exists$, $\forall$).
\end{enumerate}
Moreover, this unique operation is given inductively by
\[
  \AbsOp_i(\alpha) = \langle \alpha, \neg \alpha\rangle,\quad
  \AbsOp_i(\neg \fo) = \langle \neg \AdvOp_i(\AbsOp_i(\fo)),\ \AdvOp_i(\AbsOp_i(\fo))\rangle,
\]
\[
  \AbsOp_i(\fo \land \psi) = \AbsOp_i(\fo) \wedge \AbsOp_i(\psi),\quad
  \AbsOp_i(\fo \lor \psi) = \AbsOp_i(\fo) \vee \AbsOp_i(\psi),
\]
\[
  \AbsOp_i(\exists x.\fo) = P_i x.\,\AbsOp_i(\fo),\qquad
  \AbsOp_i(\forall x.\fo) = P_{1-i} x.\,\AbsOp_i(\fo),
\]
where $\wedge,\vee$ on the right-hand side are the componentwise game-combinators
compatible with adversarial projection.
Up to the player-rotation that swaps $0$ and $1$, this embedding is canonical.
\end{proposition}

\begin{proof}[Proof sketch]
Proceed by structural induction on the FOL formula $\fo$.

For atoms $\alpha$: by (1) we must have $\AdvOp_i(\AbsOp_i(\alpha)) = \alpha$.
By (2) we must also have $\AdvOp_{1-i}(\AbsOp_i(\alpha)) = \neg \alpha$.
But adversarial projection on terminal pairs is just projection, so $\AbsOp_i(\alpha)$ must be
$\langle \alpha, \neg \alpha\rangle$.

For Boolean connectives: by (3) we must define
$\AbsOp_i(\fo \circ \psi)$ from $\AbsOp_i(\fo)$ and $\AbsOp_i(\psi)$ using
the same constructor $\circ$. Because adversarial projection commutes with these
constructors (by definition of the calculus), (1) and (2) are preserved.

For quantifiers: consider $\exists x.\fo$.
By (1), $\AdvOp_i(\AbsOp_i(\exists x.\fo))$ must be $\exists x.\fo$.
By the specification of $\AdvOp_i$ on binders, this is possible only if
$\AbsOp_i(\exists x.\fo)$ is of the form $P_i x.\,\Phi$ with
$\AdvOp_i(\Phi)=\fo$, hence $P_i x.\,\AbsOp_i(\fo)$.
Similarly, for $\forall x.\fo$, condition (2) forces the other player’s binder:
$\AdvOp_{1-i}(\AbsOp_i(\forall x.\fo)) = \neg \forall x.\fo
= \exists x.\neg \fo$, and under the given adversarial projection convention this means
$\AbsOp_i(\forall x.\fo)$ must be $P_{1-i} x.\,\AbsOp_i(\fo)$.

Since each constructor is forced in this way, no alternative definition of
$\AbsOp_i$ can satisfy (1)–(3). Finally, choosing $i=0$ or $i=1$ just swaps the
two components, so the two embeddings are related by the role-rotation.
\end{proof}
